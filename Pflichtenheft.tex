\documentclass{article}
\usepackage{graphicx}
\usepackage{german}
\usepackage[latin1]{inputenc}

\begin{document}
\begin{titlepage}

\begin{center}
\textbf{\textsc{\LARGE Pflichtenheft}}

{\large \today}

\vspace{6cm}
\includegraphics{Bild}
Hier Logo einfuegen

\vspace{6cm}

\begin{tabular}{|c|c|c|}\hline
   Phase & Verantwortlicher & E-Mail \\ \hline\hline
   Pflichtenheft & Alina  Meixl  &  alina@meixl.de \\ \hline
   Entwurf & Viktoria Witka & witkaviktoria@freenet.de \\ \hline
   Spezifikation & Daniel Riedl & dariedl14@yahoo.de \\ \hline
   Implementation & Andreas Altenbuchner& a.andi007@gmail.com\\ \hline
   Verifikation &Patrick Kubin & kubin@fim.uni-passau.de\\ \hline
   Praesentation & w& w\\ \hline
 \end{tabular}

\end{center}

\end{titlepage}


\tableofcontents
\newpage

\section{Zielbestimmung}
Ein Online-Multiplayer Kartenspiel.

\subsection{Musskriterien}
Es gibt einen Server der das Spiel verwaltet und Clients die spielen.
\subsubsection{Server}
\begin{itemize}
	\item Clients koennen sich verbinden und eindeutigen Benutzernamen auswaehlen
	\item Lobby mit der Moeglichkeit, Spiele zu erstellen und offenen Spielen beizutreten
	\item Unterstuetzung mehrere parallel laufender Spiele
	\item Regelauswertung mit Ueberpruefung erlaubter Aktionen, Punktezaehlung, Kartenausgabe
	\item Unterstuetzung der Spiele Hearts und Wizard
	\item Chat in der Lobby
	\item Chat mit Mitspielern waehrend eines Spiels
	\item Schutz vor Cheats (Mehrfachanmeldung?)
\end{itemize}

\subsubsection{Client}
\begin{itemize}
	\item GUI
	\begin{itemize}
		\item Darstellungsfenster, das mindestens bei der Aufloesung von 1024x768 Bildpunkten benutzt werden kann
		\item Darstellung des laufenden Spiels (eigene Hand, verdeckte Hand der anderen Spieler, Ablage- und Aufnahmestapel, 			Punktestand, evtl. Zusatzinformationen)
		\item Eingabemoeglichkeiten fuer erlaubte Aktionen waehrend des Spiels (Karte ablegen, Ansagen, etc.)
		\item Die GUI muss den Benutzer sinnvoll unterstuetzen und benutzerfreundliche Eingabeelemente anbieten
		\item Fluessige Darstellung
		\item Unabhaengigkeit vom Regelwerk
		\item Beim Start Auswahl des Servers und des Benutzernamens
		\item Anzeige von offenen Spielen in der Lobby mit Moeglichkeit zum Erstellen und Beitreten
		\item Chat in der Lobby und waehrend des Spiels
	\end{itemize}
	\item Modell
	\begin{itemize}
		\item Verwaltung der Verbindung mit dem Server
		\item Verwaltung des aktuellen Spielzustands (soweit Client bekannt)
		\item Vorab-Regelauswertung zur Unterstuetzung des Nutzers (ungueltige Spielaktionen sind nicht durchfuehrbar in der 					GUI)
	\end{itemize}
\end{itemize}

\subsection{Wunschkriterien}
\begin{itemize}
	\item Weitere Regelwerke (Uno, Mau-Mau, Black Jack)
	\item Scoresystem
	\item Statistiken
	\item Mehrsprachen
	\item Veraenderbare GUI (Farben etc)
\end{itemize}

\subsection{Abgrenzungskriterien}
\begin{itemize}
	\item Beitreten eines bereits laufenden Spieles nicht moeglich.
	\item keine Persistenz ueber mehrere Sessions, keine Registrierung
	\item keine KI
\end{itemize}

\section{Produkteinsatz}
\subsection{Anwendungsbereich}
Internetspiel im Freundeskreis.
\subsection{Zielgruppe}
Personen, die gemeinsam ueber ein lokals Netzwerkoder das Internet spielen moechten. 
\subsection{Betriebsbedingungen}
Betriebsdauer ?

\section{Produktumgebung}
\subsection{Software}
	\begin{itemize}
		\item Client
		\begin{itemize}
			\item ...
		\end{itemize}
		\item Server
		\begin{itemize}
			\item ...	
			\item ...
		\end{itemize}
	\end{itemize}

\subsection{Hardware}
\begin{itemize}
		\item Client
		\begin{itemize}
			\item Internetfaehiger Rechner
		\end{itemize}
		\item Server
		\begin{itemize}
			\item Internetfaehiger Rechner	
			\item Speicherplatz
			\item Rechenleistung
		\end{itemize}
	\end{itemize}

\subsection{Orgware}
Internetverbindung.

\section{Produktfunktionen}
\subsection{Startseite}
\begin{itemize}
	\item /F040/ Auswahl vom gewuenschtem Server und Namen, danach Weiterleitung zur Lobby
\end{itemize}

\subsection{Lobby}
\begin{itemize}
	\item /F060/ Anzeige von eingeloggten Spieler
	\item /F070/ Chatten mit anderen eingeloggten Spielern auf dem Server
	\item /F080/ Anzeige offener Spiele mit Spielart und Spieleranzahl sowie die Option den Spielen beizutreten (Weiterleitung zum Wartefenster)
	\item /F090/ Option ein eigenes Spiel zu erstellen(Weiterleitung zum Erstellungsfenster)
	\item /F100W/ Moeglichkeit die Sprache zu aendern
\end{itemize}

\subsection{Erstellungsfenster}
\begin{itemize}
	\item /F120/ Auswahl vom Regelwerk, Namen des Spiels und bevorzugter Spielerzahl
	\item /F130W/ Moeglichkeit eingeloggte Spieler einzuladen
\end{itemize}

\subsection{Wartefenster}
\begin{itemize}
	\item /F150/ Anzeigen des Spieltyps, der Spieler und Spielerzahl
	\item /F160/ Ab Mindestanzahl der Spieler kann der Spielersteller des Spiel starten
	\item /F170/ Wartefenster wird nur aufgeloest wenn der Spielersteller selbst das Spiel verlaesst
\end{itemize}

\subsection{Spiel}
\begin{itemize}
	\item /F190/ Anzeige des Spiels, der eigen Karten, der verdeckten Karten der Mitspieler sowie Ablage-und Aufnahmestapel
	\item /F200/ Anzeige von Punktestand oder anderen Zusatzinformationen
	\item /F210/ Eingabemöglichkeit für regelkonforme Aktionen
	\item /F220/ Chatten mit anderen Mitspielern
	\item /F230/ Wenn einer das Spiel verlaesst wird das Spiel beendet und die anderen Mitspieler werden zur Lobby zurueckgeleitet
\end{itemize}

\section{Produktdaten}
Benutzerdaten?
Score?
Statistik?

\section{Produktleistungen}
Benutzernamen, Server, Spiel erstellen, Kartenspiel waehlen. Chat in Lobby und Spiel. Spiel spielen.

\section{Benutzungsoberflaeche}
Hier Bild einfuegen.

\section{Testszenarien}
\begin{itemize}
	\item Benutzername, Server aussuchen
	\item Spiel erstellen (Lobby)
	\item Spiel beitreten (Lobby)
	\item mehrere Spiele parallel starten (Lobby)
	\item Spiel spielen 
\end{itemize}

\section{Entwicklungsumgebung}
\subsection{Software}
\begin{itemize}
	\item LaTeX
	\item Eclipse 3.8
	\item IBM Rational Software Architect 8.0
	\item .....
\end{itemize}

\subsection{Hardware}
\begin{itemize}
	\item Rechner im CIP Pool	
	\item Private Rechner
\end{itemize}

\subsection{Orgware}
Keine.

\section{Ergaenzung}
.....
\section{Glossar}
\begin{itemize}
	\item Client: ...
	\item Server: ...
	\item Regelwerk: ...
	\item Lobby: ...
	\item ...
\end{itemize}
\end{document}
