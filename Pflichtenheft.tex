\documentclass{article}
\usepackage{graphicx}
\usepackage{german}
\usepackage[latin1]{inputenc}

\usepackage[section,toc]{glossaries}\makeglossaries

\newglossaryentry{Regelwerk}{name=Regelwerk,
	plural = Regelwerke,
	description={Regeln eines bestimmten Kartenspiels}
}
\newglossaryentry{Client}{name=Client,
	plural = Clients,
	description={Der Benutzer einer Applikation}
}
\newglossaryentry{Server}{name = Server,
	description={Rechner,der Dienste zur verfuegung stellt}
}
\newglossaryentry{Lobby}{name = Lobby,
	description={Ort, an dem Spieler ein Kartenspiel auswaehlen oder beitreten koennen}
}
\newglossaryentry{GUI}{name = GUI,
	description={Graphische Oberflaeche}
}
\newglossaryentry{Spielleiter}{name = Spielleiter,
	description={Derjenige, der in der Lobby ein neues Spiel erstellt}
}
\newglossaryentry{akkumuliert}{name = akkumuliert,
	description={Gehaeuft, nicht einzeln}
}

\begin{document}
\begin{titlepage}

\begin{center}
\textbf{\textsc{\LARGE Pflichtenheft}}

{\large \today}

\vspace{6cm}
\includegraphics{Bild}
Hier Logo einfuegen

\vspace{6cm}

\begin{tabular}{|c|c|c|}\hline
   Phase & Verantwortlicher & E-Mail \\ \hline\hline
   Pflichtenheft & Alina  Meixl  &  alina@meixl.de \\ \hline
   Entwurf & Viktoria Witka & witkaviktoria@freenet.de \\ \hline
   Spezifikation & Daniel Riedl & dariedl14@yahoo.de \\ \hline
   Implementation & Andreas Altenbuchner& a.andi007@gmail.com\\ \hline
   Verifikation &Patrick Kubin & kubin@fim.uni-passau.de\\ \hline
   Praesentation & w& w\\ \hline
 \end{tabular}

\end{center}

\end{titlepage}


\tableofcontents
\newpage

\section{Zielbestimmung}
Ein Online-Multiplayer Kartenspiel.

\subsection{Musskriterien}
Es gibt einen \gls{Server} der das Spiel verwaltet und \glspl{Client} die spielen.
\subsubsection{\gls{Server}}
\begin{itemize}
	\item \glspl{Client} koennen sich verbinden und eindeutigen Benutzernamen auswaehlen
	\item \gls{Lobby} mit der Moeglichkeit, Spiele zu erstellen und offenen Spielen beizutreten
	\item Unterstuetzung mehrere parallel laufender Spiele
	\item Regelauswertung mit Ueberpruefung erlaubter Aktionen, Punktezaehlung, Kartenausgabe
	\item Unterstuetzung der Spiele Hearts und Wizard
	\item Chat in der \gls{Lobby}
	\item Chat mit Mitspielern waehrend eines Spiels
	\item Schutz vor Cheats (Mehrfachanmeldung?)
\end{itemize}

\subsubsection{\gls{Client}}
\begin{itemize}
	\item GUI
	\begin{itemize}
		\item Darstellungsfenster, das mindestens bei der Aufloesung von 1024x768 Bildpunkten benutzt werden kann
		\item Darstellung des laufenden Spiels (eigene Hand, verdeckte Hand der anderen Spieler, Ablage- und Aufnahmestapel, 			Punktestand, evtl. Zusatzinformationen)
		\item Eingabemoeglichkeiten fuer erlaubte Aktionen waehrend des Spiels (Karte ablegen, Ansagen, etc.)
		\item Die \gls{GUI} muss den Benutzer sinnvoll unterstuetzen und benutzerfreundliche Eingabeelemente anbieten
		\item Fluessige Darstellung
		\item Unabhaengigkeit vom \gls{Regelwerk}
		\item Beim Start Auswahl des \gls{Server}s und des Benutzernamens
		\item Anzeige von offenen Spielen in der \gls{Lobby} mit Moeglichkeit zum Erstellen und Beitreten
		\item Chat in der \gls{Lobby} und waehrend des Spiels
	\end{itemize}
	\item Modell
	\begin{itemize}
		\item Verwaltung der Verbindung mit dem \gls{Server}
		\item Verwaltung des aktuellen Spielzustands (soweit \gls{Client} bekannt)
		\item Vorab-Regelauswertung zur Unterstuetzung des Nutzers (ungueltige Spielaktionen sind nicht durchfuehrbar in der 					\gls{GUI})
	\end{itemize}
\end{itemize}

\subsection{Wunschkriterien}
\begin{itemize}
	\item Weitere \glspl{Regelwerk} (Uno, Mau-Mau, Black Jack)
	\item Scoresystem
	\item Statistiken
	\item Mehrsprachen
	\item Veraenderbare \gls{GUI} (Farben etc)
\end{itemize}

\subsection{Abgrenzungskriterien}
\begin{itemize}
	\item Beitreten eines bereits laufenden Spieles nicht moeglich.
	\item keine Persistenz ueber mehrere Sessions, keine Registrierung
	\item keine KI
\end{itemize}

\section{Produkteinsatz}
\subsection{Anwendungsbereich}
Internetspiel im Freundeskreis.
\subsection{Zielgruppe}
Personen, die gemeinsam ueber ein lokales Netzwerkoder das Internet spielen moechten. 
\subsection{Betriebsbedingungen}
Betriebsdauer ?

\section{Produktumgebung}
\subsection{Software}
	\begin{itemize}
		\item \gls{Client}
		\begin{itemize}
			\item ...
		\end{itemize}
		\item \gls{Server}
		\begin{itemize}
			\item ...	
			\item ...
		\end{itemize}
	\end{itemize}

\subsection{Hardware}
\begin{itemize}
		\item \gls{Client}
		\begin{itemize}
			\item Internetfaehiger Rechner
		\end{itemize}
		\item Server
		\begin{itemize}
			\item Internetfaehiger Rechner	
			\item Speicherplatz
			\item Rechenleistung
		\end{itemize}
	\end{itemize}

\subsection{Orgware}
Internetverbindung.

\section{Produktfunktionen}
\subsection{Startseite}
\begin{itemize}
	\item /F040/ Auswahl vom gewuenschtem \gls{Server} und Namen, danach Weiterleitung zur \gls{Lobby}
	\item /F050W/ Auswahl der Sprache
\end{itemize}

\subsection{\gls{Lobby}}
\begin{itemize}
	\item /F060/ Anzeige von eingeloggten Spieler
	\item /F070/ Chatten mit anderen eingeloggten Spielern auf dem \gls{Server}
	\item /F080/ Anzeige offener Spiele mit Spielart und Spieleranzahl sowie die Option den Spielen beizutreten (Weiterleitung zum Wartefenster)
	\item /F090/ Option ein eigenes Spiel zu erstellen(Weiterleitung zum Erstellungsfenster)
	\item /F100/ Hilfe zu den Spielarten
\end{itemize}

\subsection{Erstellungsfenster}
\begin{itemize}
	\item /F120/ Auswahl vom \gls{Regelwerk} und  Namen des Spiels 
	\item /F130W/ Moeglichkeit eingeloggte Spieler einzuladen
\end{itemize}

\subsection{Passwortabfrage}
\begin{itemize}
	\item /F140/ Abfragen des vom Host gewaehlten Passworts
	\item /F142/ Zugang zum Wartefenster gewaehren
	\item /F145/ Abbrechen und zur Lobby zurueckkehren
\end{itemize}

\subsection{Wartefenster}
\begin{itemize}
	\item /F150/ Anzeigen des Spieltyps, der Spieler und Spielerzahl
	\item /F160/ Ab Mindestanzahl der Spieler kann der Spielersteller des Spiel starten
	\item /F170/ Wartefenster wird nur aufgeloest wenn der Spielersteller selbst das Spiel verlaesst
\end{itemize}

\subsection{Spiel}
\begin{itemize}
	\item /F190/ Anzeige des Spiels, der eigen Karten, der verdeckten Karten der Mitspieler sowie Ablage-und Aufnahmestapel
	\item /F200/ Anzeige von Punktestand oder anderen Zusatzinformationen
	\item /F210/ Eingabemöglichkeit für regelkonforme Aktionen
	\item /F220/ Chatten mit anderen Mitspielern
	\item /F230/ Wenn einer das Spiel verlaesst wird das Spiel beendet und die anderen Mitspieler werden zur \gls{Lobby} zurueckgeleitet
	\item /F240/ Hilfe zu dem Spiel
	\item /F250/ Auswertung bei Ende des Spiels
\end{itemize}

\section{Produktdaten}

\begin{itemize}
	\item /D010/ \gls{Lobby}daten
	 \begin{itemize}
	 	\item Spielerdaten
	 	\begin{itemize}
	 		\item Spielername(eindeutig)
	 	\end{itemize}
	 	\item Spieledaten
	 	\begin{itemize}
	 		\item Spielenamen(eindeutig)
	 		\item Spieltyp
	 		\item Anzahl an Spielern und maximale Anzahl an Spielern
	 	\end{itemize}
	 \end{itemize}
	 \item /D020/ Erstellungsdaten
	 \begin{itemize}
	 	\item \gls{Spielleiter}(eindeutig)
	 	\item Spielernamen(eindeutig)
	 	\item Spieleranzahl und maximale Spieleranzahl
	 	\item Spieltyp
	 	\item Mindestanzahl an Spielern erreicht
	 \end{itemize}
	 \item /D030/ Spieldaten
	 \begin{itemize}
	 	\item Spielname(eindeutig)
	 	\item Spieltyp
	 	\item Anzahl an Spielern
	 	\item Kartenstapel
	 	\begin{itemize}
	 		\item Anzahl verbliebener Karten
	 		\item Karten
	 		\item Ausgabe von Karten
	 	\end{itemize}
	 	\item Zugreihenfolge
	 	\item Spieler
	 	\begin{itemize}
	 		\item Name(eindeutig)
	 		\item Kartenhand
	 		\begin{itemize}
	 			\item Karten
	 			\item Anzahl
	 			\item Spielbar
	 		\end{itemize}	 
	 		\item Bedenkzeit(X minuten)		
	 	\end{itemize}
	 	\item Punktestand und Siegbedingung
	 \end{itemize}
\end{itemize}

\section{Produktleistungen}
\begin{itemize}
	\item /L040/ Einhaltung der Spielregeln gewaehrleisten
	\item /L050/ Fehlermeldungen \gls{akkumuliert} ausgeben
	\item /L060/ Verwaltung mehrerer parallel laufender Spiele
	\item /L070/ Chat und Spiel sollen fluessig laufen
	\item /L080/ Einfache und hilfreiche Bedienbarkeit
	\item /L090/ Schutz vor Cheats
	\item /L100/ Verhinderung langer Wartezeiten
	\item /L110W/ Mehrsprachigkeit unterstuetzen
\end{itemize}

\section{Benutzungsoberflaeche}
\begin{itemize}
	\item Startseite: \\ \includegraphics{GUI_images/Login}
	\item Lobby: \\ \includegraphics[scale=0.7]{GUI_images/ServerLobby}
	\item Erstellungsfenster: \\ 
		\includegraphics[scale=0.8]{GUI_images/CreateGame}
	\item Wartefenster: \\ \includegraphics[scale=0.7]{GUI_images/GameLobby}
	\item Passwortabfrage: \\ \includegraphics{GUI_images/PasswordRequest}
	\item Spiel: \\ \includegraphics[scale=0.6]{GUI_images/GameClient}
\end{itemize}

\section{Testszenarien}
\begin{itemize}
	\item Benutzername, Server aussuchen
	\item Spiel erstellen (Lobby)
	\item Spiel beitreten (Lobby)
	\item mehrere Spiele parallel starten (Lobby)
	\item Spiel spielen 
\end{itemize}

\section{Entwicklungsumgebung}
\subsection{Software}
\begin{itemize}
	\item LaTeX
	\item Eclipse 3.8
	\item IBM Rational Software Architect 8.0
	\item .....
\end{itemize}

\subsection{Hardware}
\begin{itemize}
	\item Rechner im CIP Pool	
	\item Private Rechner
\end{itemize}

\subsection{Orgware}
Keine.

\section{Ergaenzung}
..... 
\newpage
\printglossaries
\end{document}
